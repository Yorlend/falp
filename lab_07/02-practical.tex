\chapter{Практический раздел}

\section{Задание}

Запустить среду Visual Prolog5.2. Настроить утилиту TestGoal.

Запустить тестовую программу, проанализировать реакцию системы и множество ответов.

Разработать свою программу - «Телефонный справочник». Абоненты могут иметь несколько телефонов. Протестировать работу программы, используя разные вопросы.

\begin{itemize}
    \item «Телефонный справочник»: Фамилия, Noтел, Адрес – структура (Город, Улица, № дома, № кв),
    \item «Автомобили»: Фамилия владельца, Марка, Цвет, Стоимость, Номер.
\end{itemize}

Владелец может иметь несколько телефонов, автомобилей (Факты). В разных городах есть однофамильцы, в одном городе – фамилия уникальна.

Используя конъюнктивное правило и простой вопрос, обеспечить возможность поиска:

\begin{itemize}
    \item По Марке и Цвету автомобиля найти Фамилию Город, Телефон. Лишней информации не находить и не передавать.
\end{itemize}

\section{Текст программы}

\includelisting
    {prog.pl}
    {Текст программы}

\includeimage
    {res}
    {w}
    {r}
    {\linewidth}
    {Результат работы программы}
