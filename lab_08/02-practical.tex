\chapter{Практический раздел}

\section{Задание}

Необходимо Создать базу знаний \textbf{<<Собственники>>}:

\begin{itemize}
	\item \textbf{<<Телефонный справочник>>}: Фамилия, №тел, Адрес -- структура (Город, Улица, №дома, №кв);
	\item \textbf{<<Автомобили>>}: Фамилия\_владельца, Марка, Цвет, Стоимость, и др.;
	\item \textbf{<<Вкладчики банков>>}: Фамилия, Банк, счет, сумма, др.
\end{itemize}

Дополнить (минимально изменив) базу знаниями о дополнительной собственности владельца. Преобразовать знания об автомобиле к форме знаний о собственности.

Вид собственности (кроме автомобиля):

\begin{itemize}
	\item \textbf{Строение}, стоимость и другие его характеристики;
	\item \textbf{Участок}, стоимость и другие его характеристики;
	\item \textbf{Водный транспорт}, стоимость и другие его характеристики.
\end{itemize}

Описать и использовать вариантный домен: \textbf{Собственность}. Владелец может иметь, но только один объект каждого вида собственности (это касается и автомобиля), или не иметь некоторых видов собственности.

Используя конъюнктивное правило и разные формы задания одного вопроса, обеспечить возможность поиска:

\begin{enumerate}
	\item Названий всех объектов собственности заданного субъекта,
	\item Названий и стоимости всех объектов собственности заданного субъекта,
	\item * Разработать правило, позволяющее найти суммарную стоимость всех объектов собственности заданного субъекта.
\end{enumerate}

Для 2-го пункт и одной фамилии составить таблицу, отражающую конкретный порядок работы системы, с объяснениями порядка работы и особенностей использования доменов (указать конкретные Т1 и Т2 и полную подстановку на каждом шаге).

\clearpage

\section{Текст программы}

\includelisting
    {prog.pl}
    {Текст программы}

{\large\section*{Порядок поиска ответа для задания 2}}

\includeimage
    {table}
    {w}
    {r}
    {\linewidth}
    {Порядок ответа на задание 2}

