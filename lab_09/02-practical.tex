\chapter{Практический раздел}

\section{Задание}

Создать базу знаний: <<ПРЕДКИ>>, позволяющую наиболее эффективным способом (за меньшее количество шагов, что обеспечивается меньшим количеством предложений БЗ -- правил), и используя разные варианты (примеры) одного вопроса, определить (указать: какой вопрос для какого варианта):

\begin{enumerate}
	\item по имени субъекта определить всех его бабушек,
	\item по имени субъекта определить всех его дедушек,
	\item по имени субъекта определить всех его бабушек и дедушек,
	\item по имени субъекта определить его бабушку по материнской линии,
	\item по имени субъекта определить его бабушку и дедушку по материнской линии.
\end{enumerate}

Минимизировать количество правил и количество вариантов вопросов. Использовать конъюнктивные правила и простой вопрос.

Для одного из вариантов ВОПРОСА и конкретной БЗ составить таблицу, отражающую конкретный порядок работы системы.

Дополнить базу знаний правилами, позволяющими найти:

\begin{enumerate}
	\item максимум из двух чисел (с/без использования отсечения);
	\item максимум из трех чисел (с/без использования отсечения).
\end{enumerate}

Убедиться в правильности результатов.

Для каждого случая пункта 2 обосновать необходимость всех условий тела. Для одного из вариантов ВОПРОСА и каждого варианта задания 2 составить таблицу, отражающую конкретный порядок работы системы.

\clearpage

\section{Текст программы}

\includelisting
    {ancestors.pl}
    {Текст программы}

\clearpage

\section*{Порядок поиска ответа}

\includeimage
    {table}
    {f}
    {h}
    {\linewidth}
    {Порядок ответа на вопрос <<по имени субъекта определить его бабушку по материнской линии>>}

\includeimage
    {table1}
    {f}
    {h}
    {\linewidth}
    {Порядок поиска ответа для 1 варианта max}
	
\includeimage
    {table2}
    {f}
    {h}
    {\linewidth}
    {Порядок поиска ответа для 2 варианта max}

